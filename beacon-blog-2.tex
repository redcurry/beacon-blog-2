\documentclass[12pt]{article}

\title{Speciation and genetic incompatibilities in digital organisms}
\author{Carlos Anderson}

\begin{document}

\maketitle

% Double-spaced
\baselineskip 24pt


This blog post is a follow up to one I wrote last year
about my research on speciation with digital organisms.
%
One of my projects tested the hypothesis that
compensatory adaptation---an evolutionary process
in which secondary mutations can reduce
the negative effects of accumulated deleterious mutations---%
causes hybrid inviability between populations,
a form of reproductive isolation that may lead to speciation.
%
Using digital organisms, I showed that hybrids
between compensated populations inherited combinations
of deleterious and compensatory mutations
that turned out to be incompatible in the hybrid.
%
This incompatibility arose because
the full combination of deleterious and compensatory mutations
from either parent were not always present in the hybrid,
causing a mismatch between deleterious and compensatory mutations.



Over the past year, I have been working on complementing
my digital work with biological organisms, specifically, the budding yeast.
%
Yeast are unicellular fungi,
known primarily for their use in brewing and baking,
but have also been used extensively in scientific research.
%
Using common techniques in yeast genetics,
I deleted certain genes from yeast that
caused their growth rate to slow down,
simulating the effects of deleterious mutations.
%
Currently, I am allowing populations of yeast
with these deleterious mutations to evolve separately,
so that each population acquires compensatory mutations
independently of other such populations.
%
After 500 generations (they are currently at around 50),
I will hybridize different populations
and measure the growth rate of the hybrids.
%
If I find that hybrids are less fit than their parents,
I will also be able to tease apart the interactions
between deleterious or compensatory mutations
that cause incompatibilities.
%
If I find such incompatibilities,
this work would not only show that reproductive isolation
can form via compensatory adaptation,
but it would also support the utility of digital organisms
in evolutionary research.



In another study using digital organisms,
I tested whether speciation could proceed
in the face of migration between populations.
%
I also tested whether differences in environments
between populations promoted or constrained speciation.
%
I found that when the environments were different,
reproductive isolation developed even at
high migration rates of 10\%.
%
When the environments were the same, however,
even a 1\% migration prevented reproductive isolation.
%
The reason was that when the environments were the same,
any beneficial mutations in one population
that were transferred to the other population through migration
were also adaptive in the other population.
%
This caused both populations to adapt similarly,
preventing reproductive isolation between them.



Since those experiments, I have added two
variables that may affect reproductive isolation:
dimensionality and type of hybridization.
%
Dimensionality describes the number of selective pressures
in an environment.
%
It has been hypothesized that adaptation
to multiple selective pressures may increase divergence
between populations, and thus increase the probability
of genetic incompatibilities.
%
To test this hypothesis, I evolved populations
under both low and high dimensionality,
where the environments could be the same or different
between the populations (as in the previous experiments).
%
I found that, indeed, populations that adapted to a highly
dimensional environment produced hybrids that were less fit
than hybrids produced by populations adapted to few dimensions.



The other variable I added was the type of hybridization.
%
Originally, hybrids were created by exchanging
a single, contiguous region of a random size from the parental genomes.
%
Thus, hybrids often inherited complete sets of
tightly-linked co-adapted genetic regions from each parent,
hiding potential incompatibilities between populations.
%
In an attempt to expose these co-adapted genetic regions,
I implemented a hybridization method in which every site of the genome
could potentially recombine, thereby increasing the number
of recombination break points.
%
Whereas with the original hybridization method
populations that adapted to similar environments
did not form as strong reproductive isolation
as populations that adapted to different environments
(when there was no migration between populations),
the new hybridization method caused both treatments%
---same environments and different environments---%
to show similar levels of reproductive isolation.
%
These results show that genetic incompatibilities
can form even when populations adapt to the same environment.



Finally,
using digital organisms I explored a hypothesis
related to the formation of genetic incompatibilities.
%
As I have discussed,
when a population becomes geographically divided into two,
each population evolves independently of the other,
so that genetic incompatibilities may form.
%
Genetic incompatibilities may cause hybrids to be inviable,
which is a form of reproductive isolation (i.e., speciation).
%
One theoretical prediction of this process
posits that the number of genetic incompatibilities
involving two alleles should increase quadratically through time
(the so-called 'snowball effect').
%
This prediction is very difficult to test with biological organisms
because a lot of genetic manipulations would have to be performed.
%
Using digital organisms, I tested the snowball effect
and found that pairwise incompatibilities do increase quadratically
through time.
%
But I also found the presence of 'buffer' alleles in hybrids
that lessened the fitness effect of these incompatibilities,
showing that more complex interactions may explain hybrid inviability.



Overall, my research opens the exciting possibility
that speciation can form through complex genetic incompatibilities,
some of which may be due to compensatory adaptation,
regardless of the environmental differences (or lack of) between populations.
%
My next project is to examine more closely compensatory adaptation itself.
%
I will test how population size, mutation rate,
and the initial fitness effect of deleterious mutations
affect the rate of compensatory adaptation (versus reversion).
%
Knowledge of the effects of these factors on compensation
is not only relevant to my research on speciation,
but also relevant to broader topics where compensation is important,
such as the recovery of endangered species
and the fight against antibiotic resistance.

\end{document}
