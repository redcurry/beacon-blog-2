\documentclass[12pt]{article}

\title{Speciation in digital organisms}
\author{Carlos Anderson}

\begin{document}

\maketitle

% Double-spaced
\baselineskip 24pt


This blog post is a follow up to one I wrote last year
about my research on speciation with digital organisms.
%
One of my projects tested the hypothesis that
compensatory adaptation---an evolutionary process
in which secondary mutations can reduce
the negative effects of accumulated deleterious mutations---%
can lead to hybrid inviability,
a form of reproductive isolation, between populations.
%
Using digital organisms, I showed that hybrids
between compensated populations inherited combinations
of deleterious and compensatory mutations,
which turned out to be incompatible with each other
%
Over the past year, I have been working on complementing
my work with digital organisms with yeast cells.
%


Two big questions in speciation are
(1) can speciation proceed even if environments are similar to each other,
and (2) can speciation proceed when migration between populations occurs?
%
I wanted to study speciation in an artificial life system,
where my findings could be generalized to life as we don't know it.



Searching for Ph.D. programs in 2006, I found Michigan State University,
where a team of scientists and students have been developing Avida.
%
Avida is an artificial life software
designed to study questions in evolution and ecology.
%
In Avida, digital organisms consist of a sequence of instructions (or `genome')
that encodes their ability to replicate and perform computational functions.
%
The precise sequence of instructions that
allow organisms to perform functions
evolve through natural selection and genetic drift,
two evolutionary processes that occur in biological organisms.
%
With Avida, one can observe millions of generations
of evolution in a short period of time,
perform many replications, easily manipulate genomes,
and accurately record measurements like fitness and events like mutations.



One of my studies addresses whether speciation
can occur when environments are similar between populations.
%
One hypothesis is that speciation can happen by compensatory adaptation,
in which a deleterious mutation rises in frequency in a population
and is subsequently compensated by secondary mutations.
%
Imagine that two populations become divided
and each undergoes the process of compensatory adaptation.
%
If the populations were now to come into contact,
their hybrids would inherit a combination of
deleterious and compensatory mutations,
which, because they evolved independently,
may not be compatible with each other
and possibly cause inviability or sterility.
%
I found this hybrid incompatibility in Avida,
and it wasn't simply because hybrids inherited
deleterious mutations, but also because
compensatory mutations between populations were incompatible.
%
These findings show that compensatory adaptation is
one way in which speciation can occur when
the external environment does not change.


UPDATE: STARTED YEAST EXPERIMENTS



Another of my studies tests the effect of
migration between diverging populations on the probability of speciation.
%
I evolved populations that had to
adapt to a new environment while
migration between them occurred.
%
Although the environments were new,
I had treatments in which the two environments
were different from each other
and in which they were the same.
%
I found that when the environments are different,
migration does not prevent speciation from starting%
---even at 10\% migration.
%
However, when the environments are the same,
even 1\% migration prevents speciation.
%
It appears that when the environments are the same,
the population that adapts to it first
and lets an individual migrate to the other population,
effectively gives away its solution.
%
This causes both populations to adapt similarly,
preventing reproductive isolation between them.


UPDATE: MORE FINE-SCALED EXPERIMENTS
- found that at very low migration rates < 1\% isolation occurs in MO
- higher dimensional environments promotes isolation
- fine-scaled recombination exposes genetic coadapted complexes


Perhaps discuss snowball effect research



\end{document}
